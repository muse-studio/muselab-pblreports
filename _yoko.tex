\documentclass[dvipdfmx,line_length=48zw,column_gap=2zw,number_of_lines=60,baselineskip=12pt]{jlreq}
\jlreqsetup{itemization_beforeafter_space=0pt}
\makeatletter
\RenewBlockHeading{section}{1}{font={\jlreq@keepbaselineskip{\normalsize\sffamily\gtfamily}},indent=0pt,lines=1}
\RenewBlockHeading{subsection}{2}{font={\jlreq@keepbaselineskip{\normalsize\sffamily\gtfamily}},indent=0pt,lines=1}
\RenewBlockHeading{subsubsection}{3}{font={\jlreq@keepbaselineskip{\normalsize\sffamily\gtfamily}},indent=0pt,lines=1}
\makeatother
\pagestyle{empty}
% ↑この上の部分は変更しない!
% (LuaTeXを使う場合は \documentclass[dvipdfmx,... の dvipdfmx, を消す)
%%%%%%%%%%%%%%%%%%%%%%%%%%%

%%↓必要なパッケージを追加してください.
\usepackage{mathtools,amssymb}
\usepackage{newtxtext,newtxmath}

%%%%%%%%%%%%%%%%%%%%%%%%%%%

\begin{document}
\twocolumn[
{\Large
\begin{center}
%%↓タイトルを入力してください
\input{author-title}
%%%%%%%%%%%%%%%%%%%%%%%%%%%
\end{center}
}
\vspace{12pt}
\begin{flushright}
福知山公立大学情報学部 \input{author-name} \\
指導教員 橋田光代
\end{flushright}
\vspace{1\Cvs}
]

%% ここから本文を書く
\input{muselab-yoko-sample}

\bibliographystyle{muselabunsrt} % SIGMUS/EC予稿の時は ipsjunsrt にする
\bibliography{muselab-sample} % .bibファイルの名前(Zoteroからエクスポートして作る)

% \begin{thebibliography}{9}
%   \bibitem{fukuchiyama} S.~Fukuchiyama and I.~Fukuchiyama, Impact of the Fukuchiyama model and future challenges, Nature, 621, pp.~80--100, 2023.
%   \bibitem{takagi} 高木貞治,解析概論,岩波書店,1961.
%   \bibitem{empty}
% \end{thebibliography}

\end{document}
